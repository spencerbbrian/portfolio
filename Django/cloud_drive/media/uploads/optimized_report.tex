
\documentclass{article}
\usepackage{amsmath}
\title{Optimization of Data Packing and Data Center Placement}
\author{Student Name}
\date{\today}
\begin{document}
\maketitle

\section{Introduction}

This report addresses the problem of optimizing data packing on physical storage disks to minimize the number of disks used and subsequently determining the optimal locations for placing data centers to house the disks. The problem involves clients with varying data storage requirements, different service levels, and constraints related to disk capacity and data center costs.

\section{Disk Packing Problem}

The case involves 10 clients requiring storage, categorized by different service levels:
\begin{itemize}
    \item Basic: 2TB capacity (4 clients)
    \item Medium 1: 5TB capacity without redundancy (3 clients)
    \item Medium 2: 5TB capacity with redundancy (2 clients, effectively 10TB)
    \item Premium: 10TB capacity with redundancy (1 client, effectively 20TB)
\end{itemize}

Each disk has a total capacity of 25TB, and we have 10 available disks. The goal is to minimize the number of disks used.

\subsection{Decision Variables}

Let:
\begin{itemize}
    \item $x_{ij}$ represent a client’s request where $i$ refers to the client and $j$ refers to the disk allocated to the request.
    \item $z_j$ represent whether disk $j$ is used or not.
\end{itemize}

\subsection{Objective Function}

The objective is to minimize the number of disks used, formulated as:

\[
    \text{Minimize } z_1 + z_2 + \dots + z_{10}
\]

\subsection{Constraints}

Two sets of constraints are included:
\begin{itemize}
    \item The total capacity assigned to a disk must not exceed 25TB.
    \item Each client can only be assigned to one disk.
\end{itemize}

\section{Solution to Disk Packing}

Based on the solution of the continuous relaxation, the lower bound for the number of disks required is 2.52. The Mixed Integer Linear Program (MILP) gave an exact solution of 3 disks.

In the case of divisible data, the objective value remains 3, as the MILP allows for generalizing the values for a single client request.

\section{Data Center Placement Problem}

Following the disk packing optimization, the next step involves determining the best locations for placing data centers to minimize total costs. Three disks are necessary, and the disks must be placed in data centers.

Each data center has a fixed cost and a dedicated cost for hosting a disk.

\subsection{Decision Variables}

Let:
\begin{itemize}
    \item $x_{ij}$ represent whether disk $i$ is placed in data center $j$.
    \item $y_j$ represent whether data center $j$ is opened.
\end{itemize}

\subsection{Objective Function}

The objective is to minimize the total cost, including both the fixed cost for opening a data center and the dedicated cost for hosting disks, given by:

\[
\text{Minimize } 14 x_{11} + 28 x_{12} + \dots + 17 y_6
\]

\section{Solution}

The optimal solution involves placing disks 1 and 3 in data center 3, and disk 2 in data center 6, for a total cost of 73.

\section{Update Problem}

Lastly, we consider the task of minimizing the time required to dispatch an update from a source to the data centers. The optimal solution uses routers and direct paths from the source to the selected data centers.

\section{Conclusion}

This report has addressed the problem of optimizing data packing onto physical storage disks, determining data center placements, and efficiently dispatching updates in a network.

\end{document}
